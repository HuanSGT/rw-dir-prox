\documentclass{article}
\usepackage{amsmath}
\usepackage{amssymb}
\usepackage{amsthm}
\usepackage{bm}

\newtheorem*{mythm*}{Theorem}
\newtheorem*{mylem*}{Lemma}
\newtheorem*{mycor*}{Corollary}

\begin{document}

\begin{mylem*}
The expected time $r_i$ for a random walk starting at node $i$ to return to $i$ is the reciprocal of the stationary probability of $i$.
That is
\[
r_i = \frac{1}{\pi_i}.
\]
\end{mylem*}

\begin{itemize}
\item Intuitively
\begin{itemize}
\item A long walk always ends up in stationary distribution $\bm{\pi}$
\item Suppose the walk length is $T$, then the expected number of times it visits $i$ is $\pi_i \cdot T$
\item The average length between two visits is $\frac{T}{\pi_i \cdot T} = \frac{1}{\pi_i}$
\item A rigorous proof requires the Strong Law of Large Numbers
\end{itemize}
\end{itemize}

\begin{mythm*}
The probability that a random walk starting at node $i$ visits $j$ before returning to $i$, which equals $\mathrm{ep}(i\rightarrow j)$, satisfies
\[
\mathrm{ep}(i\rightarrow j)c(i,j) = \frac{1}{\pi_i},
\]
where $c(i,j)$ is the commute time between $i$ and $j$.
\end{mythm*}

\begin{itemize}
\item Proof
\begin{itemize}
\item Consider a random walk $w$ starting at $i$, and random variables
\begin{itemize}
\item $X = $ the first time $w$ returns to $i$
\item $Y = $ the first time $w$ returns to $i$ after visiting $j$
\end{itemize}
\item By definition $E(X) = \frac{1}{\pi_i}$ and $E(Y) = c(i,j)$
\item Clearly $X \leq Y$, and $\Pr\left[X = Y\right] = p \triangleq \mathrm{ep}(i\rightarrow j)$
\begin{itemize}
\item $E(Y - X) = p\cdot 0 + (1 - p)\cdot E(Y) = (1-p) c(i,j)$
\end{itemize}
\item Also $E(Y - X) = E(Y) - E(X) = c(i,j) - \frac{1}{\pi_i}$
\end{itemize}
\end{itemize}

\begin{itemize}
\item $\mathrm{ep}(i\rightarrow j) + \mathrm{ep}(j\rightarrow i) = \frac{1}{c(i,j)}\left(\frac{1}{\pi_i} + \frac{1}{\pi_j}\right)$
\item Recall that $h(i,j)$ is small whenever $\pi_j$ is large, which is bad for personalization
\begin{itemize}
\item Sarkar et al. (2008) alleviate this by restricting the length of random walk
\item Tong et al. (2007) alleviate this by reducing the dependence on stationary distribution
\end{itemize}
\end{itemize}

\end{document}
